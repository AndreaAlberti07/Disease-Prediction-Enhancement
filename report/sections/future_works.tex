\section{Limits and Future Works}
\label{sec:future_works}

In the pursuit of our final model, we navigated through a series of pivotal decisions, ranging from model selection, feature choices, 
and the intricate interplay of normalization techniques to hyperparameter tuning. These decisions, while steering us toward a robust model, 
come with inherent trade-offs, potentially leading to suboptimal outcomes. Here, we discuss some limitations in our approach and suggest 
avenues for future exploration.

\subsection{Limits}
\label{subsec:limits}

\begin{itemize}
\item \textbf{Feature Selection}: The selection of optimal features, as depicted in Figure \ref{fig:ML_operative_flow}, occurred 
before hyperparameter tuning. This sequential approach may result in the choice of an ostensibly optimal feature set, 
as both aspects are tightly linked.

\item \textbf{Hyperparameter Tuning}: The determination of the best hyperparameter combination relied on accuracy as 
the sole metric. While we employed a stratified cross-validation on a balanced dataset for a reliable accuracy estimate, 
a more comprehensive approach should encompass additional metrics such as precision, recall, and F1-score.

\item \textbf{Feature Reduction}: The feature reduction process evenly separated the four classes of symptoms and commenced 
retaining features from the class demonstrating the highest predictive power. This approach may yield suboptimal results, as a 
specific threshold might exist beyond which the predictive power of a feature diminishes. 
To better clarify this concept, let's consider the following example: suppose we have only two classes of symptoms, evenly distributed 
using the median as a threshold on the degree value. Suppose also that the predictive power of the features is the same for both classes.
In this case we cannot actually say that the degree doesn't impact the predictive power of the model. Indeed in the high degree class
we can have put lots of features with a degree not sufficiently high to become less relevant and these diseases end up
altering the result of the whole class, especially in a power law distribution context. 
A refined strategy involves employing a manual threshold for the degree value, identifying truly impactful features, 
potentially resulting in unbalanced classes.

\end{itemize}

\subsection{Future Work}

\begin{itemize}
\item \textbf{Symptoms Communities}: The features extracted from symptom communities were integrated into the model based on their 
inherent ability to capture relevant information. A potential enhancement involves leveraging this knowledge explicitly, using it as prior 
probability for the model. This entails favoring the most common diseases associated with the patient's symptoms and their communities.

\item \textbf{Multi-label Classification}: Our current approach treats diseases as independent entities. However, some diseases 
may be intricately connected. A prospective improvement entails treating diseases as a multi-label classification problem. 
For instance, the model could output the three most likely diseases instead of a singular one.

\item \textbf{Disease Complexity Analysis}: Our accuracy analysis extends to different classes of diseases based on their L1 and L2 values. 
A potential refinement involves a nuanced exploration of disease complexity, adjusting L1 and L2 thresholds to maximize accuracy 
differentials among disease classes. This approach would facilitate an in-depth analysis of diseases that pose higher prediction challenges.


\end{itemize}

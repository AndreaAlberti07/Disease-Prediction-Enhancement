
\subsection{Degree Distribution and Power Law}
% BEGIN: degree distribution and power law

% END: degree distribution and power law

\subsection{Most Important Symptoms/Diseases (4 Classes)}
% BEGIN: most important symptoms/diseases (4 classes)

% END: most important symptoms/diseases (4 classes)

\subsection{Communities}
% BEGIN: communities

% END: communities

\subsection{L2 Has No Sense, It's Right That Z-Score Low Value}
% BEGIN: L2 has no sense, it's right that z-score low value

% END: L2 has no sense, it's right that z-score low value

\subsection{Meaning of Z-Score}
% BEGIN: meaning of Z-score

% END: meaning of Z-score

\subsection{Betweenness Centrality}
% BEGIN: betweenness centrality

% END: betweenness centrality

\subsection{Clustering Coefficient}
% BEGIN: clustering coefficient
\textbf{1. Average Clustering Coefficient for the Entire Bipartite Graph (0.114):}
   - Indicates a moderate level of local clustering in the entire network, capturing the tendency of symptoms and diseases to form clusters.

\textbf{2. Average Clustering Coefficient of Diseases (0.132):}
   - Diseases exhibit a higher clustering coefficient compared to the overall graph.
   - Diseases are more interconnected with common symptoms, forming localized clusters in the network.

\textbf{3. Average Clustering Coefficient of Symptoms (0.071):}
   - Symptoms have a lower clustering coefficient compared to the overall graph.
   - Symptoms are less likely to form tightly connected clusters among themselves.

\textbf{Analysis:}
   - Diseases show a stronger tendency to share common symptoms and form clusters, contributing to the higher clustering coefficient observed for diseases.
   - Symptoms, on the other hand, exhibit a more dispersed pattern, indicating that common symptoms may not necessarily co-occur with each other at a high frequency.

\textbf{Conclusion:}
   - The network's clustering patterns suggest a structured organization, with diseases playing a central role in forming clusters based on shared symptoms.
   - The lower clustering coefficient for symptoms implies greater heterogeneity among symptoms, emphasizing the need for careful consideration of diverse symptom profiles in disease prediction.
   - These findings align with the bipartite nature of the network, highlighting the meaningful connections between diseases and symptoms that contribute to the overall clustering patterns.
% END: clustering coefficient

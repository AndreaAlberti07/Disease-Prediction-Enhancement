\section{Conclusion}

Our study successfully integrates network analysis with machine learning to enhance disease prediction models in healthcare.
By analyzing symptom-disease networks using Symptom Influence (SI) and Disease Influence (DI) indices, we uncovered
critical patterns essential for accurate disease prediction. These indices revealed diverse symptom-disease associations,
guiding the selection of features for our models.\\
Logistic Regression emerged as the most effective model, balancing accuracy and complexity, particularly when augmented with network-based features.
This model demonstrated high accuracy and managed to capture complex patterns without significant overfitting.\\
A key achievement of our study is the effective balance between feature reduction and model performance.
Focusing on significant symptoms, we reduced training time substantially while maintaining high accuracy.
This approach is especially valuable in real-world applications where computational efficiency is crucial.\\
The study, however, also recognized challenges in disease prediction, as highlighted by the analysis of specific
diseases like bladder cancer and otitis media. These cases illustrated the intricacies involved in disease classification
and the necessity for continuous model refinement.\\
In summary, this research marks a significant advancement in predictive healthcare analytics, offering a robust framework
that combines advanced network analysis with machine learning. The insights and methodologies developed hold great potential
for real-world applications, paving the way for more precise and efficient disease diagnosis and management.
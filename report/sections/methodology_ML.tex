\section{ML Model Methodology}
This section provides a comprehensive overview of the methodologies employed in the construction of the machine 
learning model. The discussion encompasses various techniques designed to handle the intricacies of model building, 
coupled with a logical flow that guides the entire process.

% ------------------- Data Preparation -------------------

\subsection{Preliminary Data Preparation}
Before delving into model development, a data preprocessing pipeline was employed to properly prepare them for the subsequent steps, facing the problems 
of class imbalance and training computational complexity.
\begin{itemize}
    \item \textbf{Random Sampling}: Given the extensive nature of hyperparameter tuning, we adopted a random sampling strategy, picking around 10\% of the dataset.
    Instead of training the models on the entire dataset for each hyperparameter combination, the random subset was used to expedite the tuning 
    phase without sacrificing model representativity.
    \item \textbf{Class Imbalance with Oversampling and Undersampling}: The dataset was highly unbalanced across its 700 disease classes.
    To mitigate this, a combination of oversampling and undersampling techniques was applied. The former was performed for minority classes,
    while the latter was applied to the majority classes, ensuring all the diseases were adequately represented during training, 
    preventing dominance and biases in the model.
\end{itemize}


% ------------------- Feature Extraction -------------------

\subsection{Feature Extraction}
% - Prepare the features and normalization
A pivotal phase in constructing a machine learning model is feature extraction. In addition to the one-hot vector representation 
of symptoms, the network analysis affords us the following features:

\begin{itemize}
    \setlength\itemsep{0.4em} % set space between items
    \item \textbf{L1 and L2 Measures}: A vector with values representing the L1 and L2 measures for each symptom.
    \item \textbf{Betweenness Centrality}: A vector with values denoting the betweenness centrality of each symptom.
    \item \textbf{Community Count}: A vector indicating the number of symptoms belonging to each community.
    \item \textbf{Community Size}: A vector replacing symptoms with the size of the community to which they belong.
\end{itemize}
\vspace{0.4cm}

Given the diverse scales of these features, normalization becomes imperative for their cohesive integration into the model 
without introducing biases. To achieve this, we opted for \textit{MaxAbs} normalization. This normalization scales each feature 
individually, ensuring that the maximal absolute value of each feature in the training set becomes 1.0, while preserving the sparsity of data.


% ------------------- Model Choice -------------------

\subsection{Model Choice}
The number of machine learning classification models available for disease prediction is vast. We decide to focus on three models
that are widely used in the literature and that are known to perform well in a variety of contexts: Logistic Regression, Random Forest, and Multilayer Perceptron (MLP).

\textbf{Logistic Regression}\\
\begin{itemize}
    \item \textbf{Strengths}: Logistic Regression's computational efficiency makes it an attractive choice for initial exploration and 
    baseline performance assessment. Its simplicity facilitates interpretability, providing insights into the impact of individual symptoms on disease prediction.
    \item \textbf{Considerations}: While efficient, Logistic Regression assumes a linear relationship between features and the 
    log-odds of the target, potentially limiting its ability to capture complex non-linear patterns.
\end{itemize}
\vspace{0.4cm}

\textbf{Random Forest}\\
\begin{itemize}
    \item \textbf{Strengths}: Random Forest is renowned for its robustness in handling large and diverse datasets, making 
    it well-suited for our expansive dataset with 700 disease classes. Moreover, Its ability to capture non-linear relationships 
    ensures that complex patterns within the symptoms' one-hot encoded features are effectively modeled.
    \item \textbf{Considerations}: The ensemble nature of Random Forest provides resilience against overfitting, a crucial factor 
    in the context of disease prediction.
\end{itemize}
\vspace{0.4cm}

\textbf{Multilayer Perceptron (MLP)}\\
\begin{itemize}
    \item \textbf{Strengths}: MLPs are adept at capturing intricate relationships in high-dimensional datasets, 
    aligning with the complexity inherent in our 300-feature symptom representation.
    \item \textbf{Considerations}: Their capacity for adapting to non-linear mappings positions MLPs as powerful 
    tools in unraveling the nuanced interactions between symptoms and diseases.
\end{itemize}



% ------------------- Operative Flow -------------------

\subsection{Operative Flow}
% - Operative Flow
	% - select best parameters for symptoms one hot only
	% - select best parameters for combination of other features (best combination is chosen with random parameters looking at the accuracy)
	% - train for each model the two version above with optimal parameters
	% - pick the best model according to accuracy
	% - train the best model with whole dataset and reduce the number of features

Once the features are ready, the core part of the model-building process can begin. 
We trained three different models: a Logistic Regression, a Random Forest, and a Multi-Layer Perceptron (MLP).

For each model, we faced the challenge of selecting both the best parameters and the most effective features. 
The interdependence between these two aspects makes the optimal approach to explore all the possible combination of features
and for each combination trying all the parameters combination. This approach is not feasible in terms of computational effort
leading us to adopt a greedy approach. We firstly split the features into two 
groups: the symptoms' one-hot vector and the remaining features. The former is used to train a base model, 
while the latter is utilized to explore the potential improvement brought by the new features.

Using Algorithm \ref{alg:feature_selection}, we determined the best feature combination for each group (symptoms and other features).
Subsequently, given the optimal feature combination, we identified the best parameter combination using Algorithm 
\ref{alg:best_selection}. Each model was then trained with the best parameters and the best features combination, 
and the model with the highest accuracy was chosen. Finally, the best overall model was trained with the entire dataset.


\begin{algorithm}[H] \small
	\caption{Feature Selection Algorithm}\label{alg:feature_selection}
	
	\begin{algorithmic}[1]
	
	\State BestFeatureComb $\gets$ EmptySet
	
	\For{each model}
	    \State CurrentModel $\gets$ EmptyModel
	    \State BestModAccuracy $\gets$ 0
	    \State Parameters $\gets$ InitializeRandomParameters
	
	    \For{i = 1 to NumberOfFeatures}
	    		\State BestAccuracy $\gets$ -1

			\For{each feature}
				\State TrainModel(CurrentModel, Parameters)
			
				\State CurrentAccuracy $\gets$ GetAccuracy(CurrentModel)
			
				\If{CurrentAccuracy $>=$ BestAccuracy}
					\State BestAccuracy $\gets$ CurrentAccuracy
					\State BestFeature $\gets$ feature
					
				\EndIf
				\State UpdateModel(CurrentModel, BestFeature)
			\EndFor
			\State FreezeFeatures(CurrentModel)
			\State ModelAccuracy[i] $\gets$ GetAccuracy(CurrentModel)
			\State FeaturesComb[i] $\gets$ GetFeat(CurrentModel)
	
	    \EndFor
	
	    \State BestComb $\gets$ FeaturesComb[Max(ModelAccuracy)]
	    \State BestFeatureComb $\gets$ BestComb $\cup$ BestFeatureComb
	
	\EndFor
	
	\State \textbf{return} BestFeatureComb
	\end{algorithmic}
\end{algorithm}

\begin{algorithm}[H] \small
	\caption{Best Model Selection Algorithm}\label{alg:best_selection}
	
	\begin{algorithmic}[1]
	
	\State CurrentAccuracy $\gets$ 0
	\For{each model}
	    
	    \State CreateGrid(Parameters)
	    \State BestParameters $\gets$ GridSearchCV(model)
	    \State TrainModel(model, BestParameters)
	    \State CurrentAccuracy $\gets$ GetAccuracy(model)

	    \If{CurrentAccuracy $>=$ BestAccuracy}
			\State BestAccuracy $\gets$ CurrentAccuracy
			\State BestModel $\gets$ model
			\State BestParameters $\gets$ Parameters
	    \EndIf

	\EndFor

	\State FullDataTrain(BestModel, BestParameters)
	\State BestAccuracy $\gets$ GetAccuracy(BestModel)
	\State ReduceFeatures(BestModel, BestParameters)
	
	\State \textbf{return} BestParameters, BestModel, BestAccuracy
	\end{algorithmic}
\end{algorithm}


At the conclusion of these procedures, we obtained the following two models:\\

\begin{itemize}
    \setlength\itemsep{0.4em} % set space between items
    \item \textbf{Symptoms Model}: The best model with the optimal parameters and the symptoms as features
    \item \textbf{Other Features Model}: The best model with the optimal parameters and the best features combination
\end{itemize}
\vspace{0.4cm}

With these models in hand, we could compare their prediction performance to evaluate whether the network 
features contributed to an accuracy improvement. Additionally, we applied the feature reduction technique discussed 
in Section \ref{sec:feature_reduction} to both models, assessing whether network features could reduce computational 
complexity without compromising accuracy.


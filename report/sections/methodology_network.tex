\section{Network Methodology}

% ------------------- Network Creation -------------------
\subsection{Network Creation}

% ------------------- Hidalgo + Significance -------------------

\subsection{Method of Reflections}

To identify influential nodes in the symptom-disease network, we introduce two indices that capture the relative importance of each actor.
The first index, referred to as the Symptom Influence (\textit{SI}) index, not only ranks symptom nodes based on their frequency (level-1)
but also considers symptom commonality.
This commonality measures whether a symptom is present in diseases affected by numerous other symptoms (level-2) or in diseases affected by only a few symptoms.\\
Conversely, the second index, known as the Disease Influence (\textit{DI}) index,
assesses the distinct symptoms related to a disease (level-1) and whether a disease exhibits symptoms that affect many other diseases (level-2).\\
In more detail, level-1 of these indices quantifies the number of symptoms associated with a disease,
while level-2 measures the commonality of these symptoms. In network theory, level-2 measures are recognized as the average nearest neighbor strength.\\
We adapt the level-\textit{N} indices following the approach of \citeauthor{Hidalgo_2007},\cite{Hidalgo_2007} and  \citeauthor{Hidalgo_2009},\cite{Hidalgo_2009}.
The level-\textit{N} indices are defined as:
\begin{equation}
    SI_{v, N} = \frac{1}{SI_{v, 1}} \sum_u W(v, u) DI_{u, N-1}
\end{equation}
\begin{equation}
    DI_{u, N} = \frac{1}{DI_{u, 1}} \sum_v W(v, u) SI_{v, N-1}
\end{equation}
\noindent
Here, $SI_{v, 1}$ and $DI_{u, 1}$ represent the level-1 indices, and $W(v,u)$ denotes the edge weight between symptom $v$ and disease $u$.
The level-1 indices are defined as follows:
\begin{equation}
    SI_{v, 1} = \sum_u W(v, u)
\end{equation}
\begin{equation}
    DI_{u, 1} = \sum_v W(v, u)
\end{equation}
\noindent
Since our network is not weighted ($W(v,u)=1$ if symptom $v$ is associated with disease $u$ and $W(v,u)=0$ otherwise),
$SI_{v,1}$ and $DI_{u,1}$ are equal to the degree of symptom $v$ and disease $u$, respectively.
\subsubsection*{Statistical Validation of SI and DI}

In our effort to identify significant nodes within the symptom-disease network,
we focus on discerning topological properties that hold statistical significance.
Our goal is to differentiate higher-order properties that are directly associated with local node features from those
that emerge from the intricate interactions among nodes.\\
Relevant studies by \citeauthor{Squartini_2011_a} \cite{Squartini_2011_a, Squartini_2011_b} and \citeauthor{Spelta_2023} \cite{Spelta_2023}
highlight how higher-order network properties naturally capture structured group interactions.
Here, a 'group' is defined as all players connected by a 'hyperlink,' representing the higher-order analog of a link.\\
The sampling of random graphs with specified properties plays a pivotal role in network analysis,
serving as fundamental null models for identifying patterns, including communities and motifs.\\
To statistically assess the significance of SI and DI, we adopt a hypothesis testing approach based on a null model.
Specifically, we posit \textbf{H0} as the hypothesis that SI and DI level-2 do not offer additional information compared to level-1,
and \textbf{H1} as the opposite.
To test these hypotheses, we generate 5000 random networks using a null model with the same level-1 properties as the original network.\\
With this ample set of null models, we assume that the distribution of SI and DI is Gaussian, leveraging the Central Limit Theorem (CLT).
For each null model, we calculate the mean ($\mu$) and standard deviation ($\sigma$) of SI and DI and compute the z-score for each SI and DI level-2,
as expressed in the following equation:

\begin{equation}
    z_{SI_{v, 2}} = \frac{SI_{v, 2} - \mu_{SI_{v, 2}}}{\sigma_{SI_{v, 2}}}
\end{equation}
\noindent
If \textbf{H0} holds true,the z-scores of SI and DI should be normally distributed with a mean of 0 and a standard deviation of 1.
Conversely, if \textbf{H1} is true, the z-scores of SI and DI should be normally distributed with a mean and standard deviation different from 0 and 1,
respectively.


% ------------------- Betweenness -------------------

\subsection{Betweenness Centrality}
The betweenness centrality of a node \(v\), as defined by \Citeauthor{Brandes_2008} \cite{Brandes_2008},
is calculated as the sum of the fraction of all-pairs shortest paths that pass through \(v\):

\begin{equation}
    c_B(v) = \sum_{s,t \in V} \frac{\sigma(s, t|v)}{\sigma(s, t)} \label{eq:betweenness}
\end{equation}

where:

\begin{itemize}
    \setlength\itemsep{0.4em} % set space between items
    \item \(V\): The set of nodes.
    \item \(\sigma(s, t)\): The number of shortest paths from node \(s\) to node \(t\).
    \item \(\sigma(s, t|v)\): The number of those shortest paths from node \(s\) to node \(t\) that pass
          through some node \(v\) other than \(s\) and \(t\).
    \item If \(s = t\), then \(\sigma(s, t) = 1\).
    \item If \(v \in \{s, t\}\), then \(\sigma(s, t|v) = 0\).
\end{itemize}

To compute the betweenness centrality, the NetworkX function \textit{nx.bipartite.betweenness\_centrality}
was utilized. This function implements the algorithm proposed by \Citeauthor{Brandes_2004} \cite{Brandes_2004},
specifically designed for bipartite graphs, and includes proper normalization for accurate results.

% ------------------- Communities Detection -------------------
\subsection{Communities Detection}
Prior to applying any community detection algorithm, two crucial steps must be performed:


\begin{itemize}
    \setlength\itemsep{1em} % set space between items
    \item \textbf{Graph Projections:} The bipartite graph needs to be projected into two separate graphs,
          one for each set of nodes. In our case, the two sets represent symptoms and diseases. To achieve this,
          the NetworkX function \textit{nx.bipartite.projected\_graph} is employed, returning the projection of
          the bipartite graph onto the specified nodes.

    \item \textbf{Compute Similarity:} The similarity between nodes needs to be computed. For our purposes,
          a co-occurrence matrix is created for each set of nodes. Taking the example of the co-occurrence matrix
          for symptoms, each entry \(s_{ij}\) represents the number of times the symptom \(i\) and the symptom \(j\)
          co-occur in the same disease.
\end{itemize}

Once the two graphs, with links weighted by node similarity, are obtained, the community detection algorithm
can be applied. We utilized the Clauset-Newman-Moore greedy modularity maximization algorithm \cite{Clauset_Newman_Moore_2004},
implemented in the NetworkX function \textit{nx.algorithms.community.greedy\_modularity\_communities}. This algorithm aims to
find the partition of the graph that maximizes modularity, defined by \Citeauthor{Newman_2006} \cite{Newman_2006} as:

\begin{equation}
    Q = \frac{1}{2m} \sum_{ij} \left[A_{ij} - \frac{k_i k_j}{2m}\right] \delta(c_i, c_j) \label{eq:modularity}
\end{equation}

where:

\begin{itemize}
    \setlength\itemsep{0.4em} % set space between items
    \item \(Q\): Modularity of the network.
    \item \(A_{ij}\): Element of the adjacency matrix representing the connection between nodes \(i\) and \(j\).
    \item \(k_i\) and \(k_j\): Degrees of nodes \(i\) and \(j\), respectively.
    \item \(m\): Total number of edges in the network.
    \item \(\delta(c_i, c_j)\): Kronecker delta function, which is 1 if \(c_i\) is equal to \(c_j\) (i.e., nodes \(i\)
          and \(j\) belong to the same community) and 0 otherwise.
    \item The sum is taken over all pairs of nodes \(i\) and \(j\).
\end{itemize}







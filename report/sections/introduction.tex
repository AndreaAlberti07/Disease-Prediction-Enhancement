\section{Introduction}

\firstword{I}{n} the dynamic healthcare landscape, a profound understanding of symptom-disease interactions is crucial for accurate diagnosis and prediction.
This report delves into this complex interplay using network analysis, integrating theoretical foundations and empirical data.
Our dual-fold objective is to provide a nuanced descriptive analysis and identify key features for predictive models.\\
The foundation of our endeavor lies in an extensive literature review on network theory and disease prediction,
ensuring a robust baseline for deeper insights. Guided by literature, we meticulously curate and analyze datasets,
preparing the groundwork for constructing meaningful networks that encapsulate symptom-disease relationships.\\
Our analysis centers on intricate network structures, utilizing bipartite models and non-weighted links to distill meaningful patterns.
We explore a spectrum of network metrics, ensuring statistical significance through null models.
Community detection algorithms reveal hidden structures and relationships between diseases, enriching our understanding.\\
As we traverse the terrain of network analysis, we introduce novel metrics inspired by the Hidalgo-Hausmann framework,
stratifying symptoms and diseases based on predictive importance. These, coupled with traditional measures like betweenness centrality,
contribute to defining features for our predictive models.\\
With a robust foundation, we transition to predictive modeling, promising enhanced performance.
Logistic regression, random forest, and multi-layer perceptron models are trained, tested,
and validated with a focus on feature importance and model improvement strategies.


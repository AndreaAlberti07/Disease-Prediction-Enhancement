\section{Introduction}

\firstword{I}{n} the dynamic field of healthcare, understanding the intricate relationships between symptoms and diseases is crucial
for precise diagnosis and predictive analytics. This study delves into these complex interactions using network analysis,
blending theoretical frameworks with empirical data. Our objectives are twofold: to unravel the complexities of these
relationships and to identify key features that enhance predictive modeling, and facilitating computational efficiency improvement.\\
Central to our analysis are complex network configurations, where we employ bipartite models and non-weighted edges to discover significant patterns.
We use a range of network metrics, comparing our results with null models to ensure statistical reliability. Our application of community detection
algorithms reveals hidden structures and relationships among diseases, enriching our understanding.\\
We leveraged metrics based on the work of \citeauthor{Hidalgo_2009}~\cite{Hidalgo_2009} and \citeauthor{Hidalgo_2007}~\cite{Hidalgo_2007},
which help classify symptoms and diseases according to their predictive strength. These metrics, along with traditional ones like betweenness centrality,
are key in characterizing our predictive models.\\
Building on this analytical groundwork, we venture into predictive modeling with the goal of exceeding current benchmarks.
In line with research by \citeauthor{Kohli}~\cite{Kohli}, \citeauthor{Singh}~\cite{Singh}, and \citeauthor{Uddin2019Dec}~\cite{Uddin2019Dec},
which highlights the efficacy of Logistic Regression, Random Forest, and Multi-Layer Perceptron algorithms in disease prediction from symptomatic data,
we focus on these models for our analysis. Furthermore we propose a graph-based approach for computational complexity reduction.\\